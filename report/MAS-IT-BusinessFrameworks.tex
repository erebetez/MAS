\documentclass[paper=a4,twoside=false,BCOR=0mm,DIV=calc,fontsize=12pt]{scrartcl}

\usepackage[automark,headsepline]{scrpage2}
% \usepackage{xunicode,fontspec,xltxtra}
\usepackage[english,german]{babel}

\usepackage[T1]{fontenc}
\usepackage[utf8x]{inputenc}

\usepackage{graphicx}
\usepackage{lastpage}
\usepackage{listings}

%Schöne Tabellen
\usepackage{booktabs}

%\usepackage{longtable}

\usepackage{multirow}

% Kein einrücken und Abstand zwischen Absätzen.
\usepackage{parskip}
\setlength{\parindent}{0cm}

% Hyperlinks und url's
\usepackage[hidelinks]{hyperref}



% \usepackage[debug]{libertine}
% \setromanfont[Mapping=tex-text]{Linux Libertine O}
% \setsansfont[Mapping=tex-text]{Linux Biolinum O}
% \setmonofont[Mapping=tex-text]{Liberation Mono}

\pagestyle{scrheadings}

\clearscrheadfoot
\ihead{\headmark}
\ohead{Seite\pagemark\ von \pageref{LastPage}}
\ifoot{Business frameworks}
\cfoot{{\includegraphics[width=1.5cm]{./img/CC3.png}}}
\ofoot{\today}

\renewcommand*{\pnumfont}{
	\normalfont\rmfamily\slshape
}

\KOMAoptions{draft=false}
\KOMAoptions{DIV=last}

\begin{document}

% --- Titelseite --- %
\begin{titlepage}
	\enlargethispage{3cm}
	\begin{raggedright}
	\begin{picture}(0,0)
		\put(-30,14){\includegraphics[width=7cm]{./img/fhnw-technik-head}}
	\end{picture}

	\vspace*{6cm}
	{\Huge\bfseries\sf
		MAS-IT\\[1.7ex]
	}
	{\Large\bfseries\sf
		Business frameworks\\[2.2ex]
	}
	{\large\bfseries\sf
		Masterthesis von\\[1.5ex]
		Etienne Rebetez\\[1.5ex]
	}
	\vspace*{1.5cm}
	{\large\bfseries\sf
		FHNW\\[1.5ex]
		Hochschule für Technik\\[1.5ex]
		Studiengang MAS-IT\\[1.5ex]
		Betreuender Dozent:\\[1.5ex]
		Dr. sc. techn. Ronald Tanner\\[1.5ex]
	}
	\vspace*{2cm}
	{\large\bfseries\sf
		Bern, \today\\
	}
	\end{raggedright}
\end{titlepage}

\newpage
% --- Zusammenfassung --- %
\section*{Summary}
Text

% --- Vorwort --- %
\section*{Vorwort}
Text
% --- Inhaltsverzeichnis --- %
\newpage
	\tableofcontents

% --- Einleitung --- %
\newpage
\section{Einleitung}
Laborprozesse. Was versteht man eigentlich darunter? 
Der stark abstrahierte Fall ist der folgende. Ein Physische Probe kommt in ein Analyselabor, bestimmte physikalische Parameter werden 
von der Probe erhoben und diese Resultate werden dem Auftraggeber bekanntgegeben.

In diesem Bericht wird ein Labor als Qualitätsicherungslabor betrachtet. Das heisst, die Prozesse und Methoden sind vorgeben und können bzw. dürfen nicht 
einfach abgeändert werden.

Traditionell wird die Probenverwaltung von einem LIMS (Labor Informations und Management System) geregelt. Traditionellerweise ist es auch so, dass
sobald die Probe welche analysiert werden muss im Labor steht, viele Schritte Manuell geschehen. Dies ist aufwendig und Fehleranfällig. 
Es darf natürlich nicht vergessen werden, dass unter Manuell auch Handarbeiten wie einwägen, Lösungen erstellen oder Analysegeräte befüllen fallen.
Für diese Aufgaben wird es noch lange Menschen geben welche diese Arbeit ausführen müssen.

Will man nun diese Laborprozesse Automatisieren, geht es vor allem um die Datenübertragung. 
Was muss gemessen werden, welche Methode soll verwendet werden,
welches qualifizierte Gerät darf verwendet werden, 
wie werden die Daten verrechnet und schlussendlich wie werden die Resultate wieder an das nächst höhere System weitergegeben.
Gleichzeitig solle das System den Laborarbeiter unterstützen und die durchgeführten Schritte sollten in alle nachvollziehbar sein.


<Geräte Software>



Menschen sind ausserordentlich gut darin Informationen 


\section{Die Laborumgebung}
Im Labor gibt es nun verschiedenste Methoden welche unterschiedliche Historische Ursprünge haben. So ist eine Chemische Titration ein
vollkommen anderes Konzept als z.B eine Biochemische Elisa Analyse.
Dies ergibt ein äusserst Heterogene Landschaft an Analysegeräten mit und ohne Software. Resultate oder Rohdaten liegen in unterschiedlichsten Formaten und Layouts vor.





\section{Konzept}
Um eine Automatisierung in diesem Heterogenen Umfeld realisieren zu können, braucht es eine Engine die einerseits den Status der einzelnen
Abläufe kennt und auch weiss welcher Businessprozess als nächster aufgerufen werden soll. Diese Statemachien oder Sheduler währe somit das
Herzstück der Automatisierung. Häufig fällt auch der Begriff SOA (Service Oriented Architecture).

Das Labor ist lange nicht der einzige Bereich welcher eine Automatisierung verlangt. Andere Branchen habe grundsätzlich das selbe
Bedürfnis. Daher gibt es schon eine reihe von Produkten welche diese Statemachienen bereits anbieten.

Die Konfiguration dieser Engines geschieht über standardisierte XML Files. Dabei haben sich zwei Beschreibungssprachen etabliert:
\begin{itemize}
 \item BPEL
 \item BPMN
\end{itemize}

Beide sind offene Standards vom OMG \cite{omg}.



\section{BPEL}
Die BPEL (Business Process Execution Language) Beschreibungsprache kommt aus der Informatikwelt. Diese wird hat Beschreibungen zu
Schleifen, Bedingugnen und anderes Konstrukte die man aus der Programmierung kennt. 


\section{BPMN}
BPMN (Business Process Model and Notation) kommt aus dem Business und war gedacht erstmals auf dem Papier die Prozesse einheitlich
beschreiben zu können. Grob gesagt, handelt es sich um Flowcharts. 
Mit der Version 2 vom BPMN können diese Flowcharts nun auch programmatisch ausgeführt werden. 




\section{Engines}
Die Engine ist nun eine Implementierung einer solchen Business Prozess Beschreibung. Die Engines erfüllen nun die Logik. 
\begin{itemize}
  \item Welche Prozesse gibt es?
  \item Welche sind aktiv und in welchem zustand?
  \item Der Prozess wird korrekt weiter geführt, auch wenn der Server Abstürzen würde.
  \item Was passiert mit einer neuen Version eines Prozessen und so weiter.                      
\end{itemize}

Das sind alles relativ komplexe aufgaben, welche man im Normalfall nicht selber neu erfinden möchte.

Es folgt nun ein verglich verschiedener Engines. Die getroffene Auswahl ist weit weg von vollständig.



\subsection{activVOS}
Active Endpoints, Inc. 

java. 
proprietär (setzt aber auf offenen standards)
eclipse plugin.
Viele Guis
Hat zum Beispiel ein Report designer.

Preise?

Dokumentation:
Diverse Videos und demos

\subsection{biztalk}
Microsoft
.NET
proprietär
nur bpel


Preise?

Dokumentation:


\subsection{Apache ODE}
apache fundation
java
open source
nur bpel engine



\subsection{activiti}
java
opensource
nur bpmn engine
open source
schlank und aufgeräumt



\subsection{Übersicht}




\section{Technologie Wahl}



\section{Datennormalisierung}





\addcontentsline{toc}{section}{Literaturverzeichnis}
\begin{thebibliography}{99}

\bibitem{activiti} \url{http://www.activiti.org/}

\bibitem{omg} \url{http://www.omg.org/}




\end{thebibliography}

\end{document}

